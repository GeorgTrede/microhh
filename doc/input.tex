% The design of this document has been borrowed from the DALES manual.
\documentclass[a4paper,8pt, twocolumn]{extarticle}
\usepackage[twoside, bindingoffset=0cm, paper=a4paper, left=0.6cm, right=0.6cm, top=0.6cm, bottom=0.2cm, landscape]{geometry}
\usepackage{amsmath}
\usepackage{supertabular}
\usepackage{longtable}

%% Fonts
%\usepackage{pxfonts}
\usepackage{fourier}
\usepackage[T1]{fontenc}

% Column widths:
\def \wname{1.8cm}
\def \wdesc{7.2cm}

\begin{document}
\section*{\huge MicroHH 1.0 cheat sheet}

\subsection*{[advec] Advection}
\tablefirsthead{\hline NAME & DEFAULT VALUE & OPTIONS & DESCRIPTION \\ \hline}
\tablehead{\multicolumn{4}{l}{\small\sl ... continued from previous page} \\  NAME & DEFAULT VALUE & OPTIONS & DESCRIPTION \\ \hline}
\tabletail{\hline \multicolumn{4}{l}{\small\sl Continued on next page ...} \\} 
\tablelasttail{\hline}
\begin{supertabular}{|p{\wname} c c p{\wdesc}|}
swadvec       & swspatialorder       & 0   & disable advection \\ 
              &                      & 2   & 2nd-order advection \\
              &                      & 2i4 & 2nd-order advection with 4th-order interpolation \\  
              &                      & 4   & 4th-order advection (high accuracy) \\
              &                      & 4m  & 4th-order advection (energy conserving) \\
cflmax        & 1.0                  &     & \\
\end{supertabular}

\subsection*{[boundary] Boundary conditions}
\tablefirsthead{\hline NAME & DEFAULT VALUE & OPTIONS & DESCRIPTION \\ \hline}
\tablehead{\multicolumn{4}{l}{\small\sl ... continued from previous page} \\  \hline NAME & DEFAULT VALUE & OPTIONS & DESCRIPTION \\ \hline}
\tabletail{\hline \multicolumn{4}{l}{\small\sl Continued on next page ...} \\} 
\tablelasttail{\hline}
\begin{supertabular}{| p{\wname} c c p{\wdesc} |}
swboundary    & default & default   & fully resolved boundaries (smooth wall) \\
              &         & surface   & MOST based wall model \\
mbcbot        & n/a     & noslip    & no-slip bottom boundary condition \\
              &         & freeslip  & free-slip bottom boundary condition \\
              &         & ustar     & fixed ustar bottom boundary condition \\
mbctop        & n/a     & noslip    & no-slip top boundary condition \\
              &         & freeslip  & free-slip top boundary condition\\
sbcbot[]      & n/a     & dirichlet & fixed bottom value boundary condition \\
              &         & neumann   & fixed bottom gradient (only valid for \textit{default}) \\
              &         & flux      & fixed bottom flux boundary condition \\
sbcbot[]      & n/a     & dirichlet & fixed top value boundary condition \\
              &         & neumann   & fixed top gradient (only valid for \textit{default}) \\
              &         & flux      & fixed top flux boundary condition \\
sbot[]        & n/a     &           & value of the bottom boundary condition\\
stop[]        & n/a     &           & value of the top boundary condition \\
\hline \multicolumn{4}{l}{Only for swboundary = \textit{surface}:} \\ \hline
z0m           & n/a     &           & roughness length of momentum [m] \\
z0h           & n/a     &           & roughness length of scalars [m]\\
ustar         & n/a     &           & value of the friction velocity [m~s$^{-1}$]\\
\end{supertabular}

\subsection*{[buffer] Buffer layer}
\tablefirsthead{\hline NAME & DEFAULT VALUE & OPTIONS & DESCRIPTION \\ \hline}
\tablehead{\multicolumn{4}{l}{\small\sl ... continued from previous page} \\  \hline NAME & DEFAULT VALUE & OPTIONS & DESCRIPTION \\ \hline}
\tabletail{\hline \multicolumn{4}{l}{\small\sl Continued on next page ...} \\} 
\tablelasttail{\hline}
\begin{supertabular}{|p{\wname} c c p{\wdesc}|}
swbuffer & 0   & 0 & disable buffer layer at the top of the domain \\
         &     & 1 & enable buffer layer at the top of the domain \\
zstart   & n/a &   & starting height for buffer zone [m]\\
sigma    & n/a &   & damping time scale of the buffer layer [s$^{-1}$]\\
beta     & 2.  &   & exponent of the damping increase with height [-] \\
\end{supertabular}

\subsection*{[cross] Cross-section}
\tablefirsthead{\hline NAME & DEFAULT VALUE & OPTIONS & DESCRIPTION \\ \hline}
\tablehead{\multicolumn{4}{l}{\small\sl ... continued from previous page} \\  \hline NAME & DEFAULT VALUE & OPTIONS & DESCRIPTION \\ \hline}
\tabletail{\hline \multicolumn{4}{l}{\small\sl Continued on next page ...} \\} 
\tablelasttail{\hline}
\begin{supertabular}{|p{\wname} c c p{\wdesc}|}
swcross       & 0     & 0 & disable cross sections \\
              &       & 1 & enable cross sections \\ 
sampletime    & n/a   &   & sampling time step [s] \\
xz            & empty &   & list of y locations at which xz-crosssection are taken \\
xy            & empty &   & list of z locations at which xy-crosssection are taken \\
crosslist     & empty &   & list of cross-section variables \\
\end{supertabular}

\newpage

\subsection*{[diff] Diffusion}
\tablefirsthead{\hline NAME & DEFAULT VALUE & OPTIONS & DESCRIPTION \\ \hline}
\tablehead{\multicolumn{4}{l}{\small\sl ... continued from previous page} \\  \hline NAME & DEFAULT VALUE & OPTIONS & DESCRIPTION \\ \hline}
\tabletail{\hline \multicolumn{4}{l}{\small\sl Continued on next page ...} \\} 
\tablelasttail{\hline}
\begin{supertabular}{|p{\wname} c c p{\wdesc}|}
swdiff        & swspatialorder       & 0     & disable diffusion \\
              &                      & 2     & 2nd-order diffusion \\
              &                      & 4     & 4th-order diffusion \\
              &                      & smag2 & 2nd-order Smagorinsky eddy diffusion \\
dnmax         & 0.4                  &       & maximum diffusion number for numerical scheme \\
\hline \multicolumn{4}{l}{Only for swdiff = \textit{smag2}:} \\ \hline
cs            & 0.23                 &       & Smagorinsky constant \\
tPr           & 1./3.                &       & turbulent Prandtl number \\
\end{supertabular}

\subsection*{[dump] 3D output}
\tablefirsthead{\hline NAME & DEFAULT VALUE & OPTIONS & DESCRIPTION \\ \hline}
\tablehead{\multicolumn{4}{l}{\small\sl ... continued from previous page} \\  \hline NAME & DEFAULT VALUE & OPTIONS & DESCRIPTION \\ \hline}
\tabletail{\hline \multicolumn{4}{l}{\small\sl Continued on next page ...} \\} 
\tablelasttail{\hline}
\begin{supertabular}{|p{\wname} c c p{\wdesc}|}
swdump        & 0     & 0 & disable writing 3d diagnostic fields \\
              &       & 1 & enable writing 3d diagnostic fields \\ 
sampletime    & n/a   &   & sampling time step [s] \\
dumplist      & empty &   & list of diagnostic 3D fields \\
\end{supertabular}

\subsection*{[fields] Fields}
\tablefirsthead{\hline NAME & DEFAULT VALUE & OPTIONS & DESCRIPTION \\ \hline}
\tablehead{\multicolumn{4}{l}{\small\sl ... continued from previous page} \\  \hline NAME & DEFAULT VALUE & OPTIONS & DESCRIPTION \\ \hline}
\tabletail{\hline \multicolumn{4}{l}{\small\sl Continued on next page ...} \\} 
\tablelasttail{\hline}
\begin{supertabular}{|p{\wname} c c p{\wdesc}|}
slist         & empty &  & list of passive scalars \\
visc          & n/a   &  & viscosity [m$^2$ s$^{-1}$] \\
svisc[]       & n/a   &  & diffusivity of scalars [m$^2$ s$^{-1}$] \\
rndseed       & 2     &  & seed of the randomnizer \\
rndamp[]      & 0.    &  & amplitude of random perturbations [variable unit] \\
rndz          & 0.    &  & maximum height of perturbations [m] \\
rndexp        & 2.    &  & exponent of decay of perturbation \\
vortexnpair   & 0     &  & number of rotating vortex pairs \\
vortexamp     & 1.e-3 &  & amplitude of vortex pairs \\
vortexaxis    & x     &  & axis around which the vortices are evolving \\
\end{supertabular}

\subsection*{[force] Large scale forcings}
\tablefirsthead{\hline NAME & DEFAULT VALUE & OPTIONS & DESCRIPTION \\ \hline}
\tablehead{\multicolumn{4}{l}{\small\sl ... continued from previous page} \\  \hline NAME & DEFAULT VALUE & OPTIONS & DESCRIPTION \\ \hline}
\tabletail{\hline \multicolumn{4}{l}{\small\sl Continued on next page ...} \\} 
\tablelasttail{\hline}
\begin{supertabular}{|p{\wname} c c p{\wdesc}|}
swlspres      & 0     & 0     & disable large scale pressure forcing \\
              &       & geo   & use geostrophic wind as large scale pressure force \\
              &       & uflux & fix the mean flow velocity in the x-direction \\
swls          & 0     & 0     & disable large scale source/sink \\
              &       & 1     & enable large scale source/sink \\
lslist        & empty &       & list of prognostic variables having large scale source/sink \\
swwls         & 0     & 0     & disable large scale vertical velocity \\
              &       & 1     & enable large scale vertical velocity \\
fc            & n/a   &       & coriolis parameter [s$^{-1}$] \\
uflux         & n/a   &       & mean flow velocity [m~s$^{-1}$] \\
\end{supertabular}

\subsection*{[grid] Grid}
\tablefirsthead{\hline NAME & DEFAULT VALUE & OPTIONS & DESCRIPTION \\ \hline}
\tablehead{\multicolumn{4}{l}{\small\sl ... continued from previous page} \\  \hline NAME & DEFAULT VALUE & OPTIONS & DESCRIPTION \\ \hline}
\tabletail{\hline \multicolumn{4}{l}{\small\sl Continued on next page ...} \\} 
\tablelasttail{\hline}
\begin{supertabular}{|p{\wname} c c p{\wdesc}|}
xsize          & n/a   &   & x-size of domain [m]\\
ysize          & n/a   &   & y-size of domain [m]\\
zsize          & n/a   &   & z-size of domain [m]\\
itot           & n/a   &   & number of grid points in x-direction \\
jtot           & n/a   &   & number of grid points in y-direction \\
ktot           & n/a   &   & number of grid points in z-direction \\
swspatialorder & n/a   & 2 & 2nd-order spatial discretization \\
               &       & 4 & 4th-order spatial discretization \\
swbasestate    & n/a   & boussinesq & constant density and reference temperature \\
               &       & anelastic  & anelastic approximation (Bannon, 1996) \\
utrans         & 0.    &   & translation velocity in x-direction [m s$^{-1}$] \\
vtrans         & 0.    &   & translation velocity in y-direction [m s$^{-1}$] \\
\end{supertabular}

\subsection*{[mpi] Communication}
\tablefirsthead{\hline NAME & DEFAULT VALUE & OPTIONS & DESCRIPTION \\ \hline}
\tablehead{\multicolumn{4}{l}{\small\sl ... continued from previous page} \\  \hline NAME & DEFAULT VALUE & OPTIONS & DESCRIPTION \\ \hline}
\tabletail{\hline \multicolumn{4}{l}{\small\sl Continued on next page ...} \\} 
\tablelasttail{\hline}
\begin{supertabular}{|p{\wname} c c p{\wdesc}|}
npx           & 1     & & number of processors in x-direction \\
npy           & 1     & & number of processors in y-direction \\
\end{supertabular}

\subsection*{[pres] Pressure}
\tablefirsthead{\hline NAME & DEFAULT VALUE & OPTIONS & DESCRIPTION \\ \hline}
\tablehead{\multicolumn{4}{l}{\small\sl ... continued from previous page} \\  \hline NAME & DEFAULT VALUE & OPTIONS & DESCRIPTION \\ \hline}
\tabletail{\hline \multicolumn{4}{l}{\small\sl Continued on next page ...} \\} 
\tablelasttail{\hline}
\begin{supertabular}{|p{\wname} c c p{\wdesc}|}
swpres        & swspatialorder        & 0 & disable pressure solver \\
              &                       & 2 & 2nd-order pressure solver (tridiagonal solver) \\
              &                       & 4 & 4th-order pressure solver (heptadiagonal solver) \\
\end{supertabular}

\subsection*{[stat] Statistics}
\tablefirsthead{\hline NAME & DEFAULT VALUE & OPTIONS & DESCRIPTION \\ \hline}
\tablehead{\multicolumn{4}{l}{\small\sl ... continued from previous page} \\  \hline NAME & DEFAULT VALUE & OPTIONS & DESCRIPTION \\ \hline}
\tabletail{\hline \multicolumn{4}{l}{\small\sl Continued on next page ...} \\} 
\tablelasttail{\hline}
\begin{supertabular}{|p{\wname} c c p{\wdesc}|}
swstats       & 0     & 0   & disable statistics \\
sampletime    & n/a   &     & sampling time step [s] \\
\end{supertabular}

\subsection*{[thermo] Thermodynamics}
\tablefirsthead{\hline NAME & DEFAULT VALUE & OPTIONS & DESCRIPTION \\ \hline}
\tablehead{\multicolumn{4}{l}{\small\sl ... continued from previous page} \\  \hline NAME & DEFAULT VALUE & OPTIONS & DESCRIPTION \\ \hline}
\tabletail{\hline \multicolumn{4}{l}{\small\sl Continued on next page ...} \\} 
\tablelasttail{\hline}
\begin{supertabular}{|p{\wname} c c p{\wdesc}|}
swthermo      & 0         & 0     & disable thermodynamics \\
              &           & dry   & dry thermodynamics \\
              &           & moist & moist thermodynamics \\
              &           & buoy  & buoyancy thermodynamics \\
              &           & buoy\_slope  & buoyancy thermodynamics including slope\\
thvref0       & n/a       &       & reference virtual potential temperature [K] (moist Boussinesq) \\
thref0        & n/a       &       & reference potential temperature [K] (dry Boussinesq) \\
ps            & n/a       &       & surface pressure [Pa] \\
swupdatebasestate & n/a   & 0     & use initial hydrostatic pressure in $q_l$ calculation \\
              &           & 1     & update hydrostatic pressure in $q_l$ calculation \\         
\end{supertabular}

\newpage

\subsection*{[timeloop] Time}
\tablefirsthead{\hline NAME & DEFAULT VALUE & OPTIONS & DESCRIPTION \\ \hline}
\tablehead{\multicolumn{4}{l}{\small\sl ... continued from previous page} \\  \hline NAME & DEFAULT VALUE & OPTIONS & DESCRIPTION \\ \hline}
\tabletail{\hline \multicolumn{4}{l}{\small\sl Continued on next page ...} \\} 
\tablelasttail{\hline}
\begin{supertabular}{|p{\wname} c c p{\wdesc}|}
starttime     & n/a   &       & start time of simulation [s] \\
endtime       & n/a   &       & end time of simulation [s] \\
savetime      & n/a   &       & interval for saving restart files [s] \\
postproctime  & n/a   &       & time step of postprocessing procedure \\
adaptivestep  & true  & true  & enable adaptive time stepping \\
              &       & false & disable adaptive time stepping \\
dt            & 0.1   &       & time step [s] (only valid if adaptivestep = false) \\
dtmax         & dbig  &       & maximum time step [s] \\
rkorder       & 3     & 3     & Runge-Kutta 3rd-order accuracy, 3 steps \\
              &       & 4     & Runge-Kutta 4th-order accuracy, 5 steps \\
outputiter    & 10    &       & frequency of diagnostic output to $<$casename$>$.out \\
iotimeprec    & 0     &       & precision of saving of time in 10-power (i.e. -1 = 0.1, etc.) \\
\end{supertabular}

\end{document}
